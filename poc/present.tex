\catcode`\@=11


\parindent0pt		% Indentation probably not good for presentation
\parskip0pt		% Neither is automatic extra space between paragraphs

%
% Set some fonts
%
\font\titlefont=cmssbx10 at 20pt % Define fonts for title of presentation
\font\normalfont=cmss12          % ordinary text
\font\slidetitlefont=cmssbx10    % title of a slide
\font\it=cmti12
%
\baselineskip=15pt
%			%% Below fonts for math are adapted to 12pt size
\font\tfont=cmr12
\font\sfont=cmr9
\font\ssfont=cmr7
\font\tifont=cmmi12
\font\sifont=cmmi9
\font\ssifont=cmmi7
\font\tsyfont=cmsy10 at 12pt
\font\ssyfont=cmsy9
\font\sssyfont=cmsy7
\font\texfont=cmex10 at 12 pt
\font\sexfont=cmex9
\font\ssexfont=cmex7
%
\textfont0=\tfont
\scriptfont0=\sfont
\scriptscriptfont0=\ssfont
\textfont1=\tifont
\scriptfont1=\sifont
\scriptscriptfont1=\ssifont
\textfont2=\tsyfont
\scriptfont2=\ssyfont
\scriptscriptfont2=\sssyfont
\textfont3=\texfont
\scriptfont3=\sexfont
\scriptscriptfont3=\ssexfont
%

\def\NewSlide{\vfill\eject}                      % Begin a new slide; slide number increases

\def\SlideTitle#1{%                              % Set the slide title, show it as the headline
\global\headline{%                               % of the slide in slidetitlefont, with a
\vbox to 5mm{%                                   % horizontal line separating the headline from
\hbox to \hsize{\slidetitlefont #1}%             % the main body of the slide
\vskip3pt plus 1fil minus 1fil
\hrule height 1pt depth 0pt
}}}

%
% Includes an image. The image file name is #2, and #1 is a specification of the dimension of the
% image (will be scaled accordingly) in the usual TeX syntax (width, height, depth).
%
\def\image[#1]#2{%
\pdfximage #1 {#2}\pdfrefximage\pdflastximage}

\pdfpagewidth=144mm
\pdfpageheight=90mm
\vsize=67mm
\hsize=134mm
\hoffset=-20.4mm
\voffset=-12.4mm

\def\rm{\fam=0} %% Added 14. September 2010. Use roman font in math.

\catcode`\@=12
